%===================================== CHAP 4 =================================

\chapter{Four Point Probe}
    The four-probe technique is a method that allows for sensitive resistivity measurements by eliminating the resistance of the measurement probes and contacts from the measurement. In a two point resistivity measurement the resistance is the sum of the resistance of the leads, the resistance of the contacts and the resistance of the sample. When the resistance of the sample is not large compared to the other resistances the method is no longer useful. The four-probe methods uses four contacts to the sample, two outer probes supply current to the sample, and the voltage is measured across the two inner probes. In the ideal case, no current flows through the voltage probes, leading to no voltage drop across the contacts and leads, and thus an accurate measurement of the resistance of the sample. Typical four-probes are designed to measure semicondutor wafers and thin films and consist of four co linear Tungsten probes with a constant spacing. 
    
 \subsection{Sample Geometry}
    
\cleardoublepage